\begin{abstract}
%The INTO-CPS Application is a common interface used to access different artefacts in the development of a Cyber-%Physical System (CPS) making use of a collection of different independent tools connected in an open INTO-CPS %platform. For the configuration of the composition of Functional Mockup Units (FMUs) the INTO-CPS Application %has traditionally imported this information from a SysML model made using the Modelio tool. The contribution %presented in this paper is an ability to make an ability for using such a graphical configuration directly %inside the INTO-CPS Application (replacing the current less intuitive possibility for doing this).


The INTO-CPS Application is a common interface used to access and create different artefacts in the development of a Cyber-Physical System (CPS) making use of a collection of tools centred around the Functional Mockup Interface (FMI). For the configuration of the composition and adaptation of Functional Mockup Units (FMUs) the application currently imports this information as a multi-model from a SysML model made using the Modelio tool. The contribution presented in this paper is the addition of a unified graphical editor to the INTO-CPS application which eliminates the need for external tools. This is accomplished by using a standard called System Structure and Parameterisation (SSP) complementing FMI, since it is more expressive than the current multi-model representation. SSP enables the description of co-simulation scenarios in a graphical form including semantic adaptation of FMUs using SSP's extension capabilities.

%Additionally, this editor extends the existing functionality by allowing several types of semantic adaptations to be applied to one or more FMUs. Furthermore, the paper defines how these concepts can be represented under the recently published System Structure and Parameterisation (SSP) standard accompanying FMI.
\end{abstract}